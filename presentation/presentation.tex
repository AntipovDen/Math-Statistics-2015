\documentclass[hyperref=unicode,graphics=pdflatex,13pt]{beamer}

\mode<presentation>
{
  \usetheme{default}
%  \usecolortheme{crane}
  \setbeamercovered{invisible}
  \setbeamertemplate{footline}{\hfill\insertframenumber/\inserttotalframenumber}
  \setbeamersize{text margin left=0.4cm,sidebar width left=0cm}
}

\graphicspath{{pic/}}

\usepackage[utf8]{inputenc}
\usepackage[T2A]{fontenc}
\usepackage[english, russian]{babel}
\usepackage{times}
\usepackage{color, colortbl}
\definecolor{res}{RGB}{255, 215, 0}
\definecolor{olivegreen}{RGB}{38, 141, 38}
\usepackage{xspace}
\usepackage{tabularx}

\beamertemplatenavigationsymbolsempty
%\setbeamerfont{page number in head/foot}{size=\HUGE}


\title[Dishonest Casino Problem\ldots]{Dishonest Casino Problem
}

\author[\mbox{V. Mironovich, D. Antipov, V. Volochay}]
{V. Mironovich \and D. Antipov \and V. Volochay}
\institute{\inst{1}ITMO University
}
\date[June 4]{June 4, 2015}

\AtBeginSection[] {
  \begin{frame}<beamer>{}
    \tableofcontents[currentsection]
  \end{frame}
}

\begin{document}

\begin{frame}[noframenumbering,plain]
  \titlepage
\end{frame}

\begin{frame}[noframenumbering,plain]
  \tableofcontents
\end{frame}

\section{Introduction}
\subsection{Dishonest Casino Problem}
\begin{frame}{Dishonest Casino Problem}
\begin{itemize}
   \item Two sources of random values
   \item They switch in some unknown moments of time
   \item They generate a sequence
   \item The problem is to detect the switches
\end{itemize}
\end{frame} 

\subsection{Our first approaches}
\begin{frame}{Main idea}
\begin{itemize}
   \item Generate a binary vector $S$ and assume:
   \begin{itemize}
   \item $S_t = 0$, if $X_t$ was given by first source
   \item $S_t = 1$, if $X_t$ was given by second source
   \end{itemize}
   \item Optimize this vector using evolutionary algorithms
   \item What fitness should we choose?
\end{itemize}
\end{frame}

\begin{frame}{First decisions}
\begin{itemize}
   \item Minimize the avarage deviation of $p(S_i=s | X_i = x)$ calculated by the Bayes theorem from this probabilities calculated by assumption vector
   \item Maximize the Bhattacharyya distance beetween the sourse distributions:
   $$D_B(p, q) = -\ln{\sum_{\omega \in \Omega} \sqrt{p(\omega)q(\omega)}}$$
   \item %TODO: Tut Vika rasskazhet pro Kolmagorova
\end{itemize}
\end{frame}

\begin{frame}{First fails}
\begin{itemize}
   \item This methods found the optimal solutions when $p(X_i = x | S_i = 0) = 0 \oplus p(X_i = x | S_i = 1) = 0$
   \item Assume that $p(X_i = x | S_i = s) \ge q > 0$. $q$ may be $\frac{1}{len(X)}$
   \item New approach found solutions with more optimal fitness functions that had a poor correlation with the real solution
   \item We can assume that we know the fitness of real solution and optimize the absolute difference of fitnesses
\end{itemize}
\end{frame}

\section{Solution}
\subsection{Sliding windows}
\begin{frame}{One sliding window}
\begin{itemize}
   \item Choose the length of the window equal to the expected length of non-switching subsequence of $S$
   \item We can calculate the expectation for every position of window of the in-window sequence
   \item It gives us nothing even after smoothing the expectations:
   %TODO: some pictures here 
\end{itemize}
\end{frame}

\begin{frame}{One sliding window}
\begin{itemize}
   \item Basing on calculated expectations we can try to assume that expectations of sources are the max and min expectation
   \item With this assumption we can try to find probabilities of each source on every step
   \item %TODO check the probability again or say that it is useless
\end{itemize}
\end{frame}

\begin{frame}{Two Sliding Windows}
\begin{itemize}
   \item Two sliding windows without interval between them
   \item Check the difference between them:
   \begin{itemize}
      \item difference of expectations
      \item Bhattacharyya distance
   \end{itemize}
   \item Nothing of it correlates with the real $S$
\end{itemize}

\end{frame}

\end{document}

